\documentclass[onecolumn,10pt]{journal}
\usepackage{fullpage}
\renewcommand{\baselinestretch}{2}
\author{Tarun Rajendran and Yiyi Wang}
\title{Partial Redundancy Elimination via Lazy Code Motion}
\begin{document}
\maketitle

\section*{Problem Statement}

Partial Redundancy Elimination is a well used transformation to optimize code. While a powerful optimization by itself, previous naive implementations have had limitations and small benefit. Amongst the various implementations of Partial Redundancy Elimination, this implementations seeks to improve the effectiveness of current partial redundancy elimination techniques. Improvement to this fundamental algorithm may yield interesting results and lead to future work in this area. \\Not Done

\section*{Existing Work}

\section*{Algorithm Overview}

\section*{Implementation Details}

\section*{Experimental Results}

\section*{References}

\section*{Appendix}

\section*{Running Instructions}

\section*{Example}

\end{document}


